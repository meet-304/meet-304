\documentclass[a4paper,11pt,twocolumn]{article}
\usepackage[a4paper,left=1.5cm,right=1cm,top=2cm,bottom=2cm]{geometry}
\usepackage{setspace}
\usepackage{gensymb}
\usepackage{caption}
\usepackage{graphicx}
\usepackage{tabularx}
\usepackage{lmodern}
\usepackage{watermark} 
\usepackage{lipsum}
\usepackage{xcolor}
\usepackage{listings}
\usepackage{graphicx}
\usepackage{enumitem}
\usepackage{mathtools}
\usepackage{titlesec}
\usepackage[utf8]{inputenc}
\usepackage{fontenc}
\usepackage{harvard}
\usepackage{amsfonts}
\usepackage{tikz}
\usepackage{}
\graphicspath{{/storage/emulated/0/Download/FWC/Latex/figs}}
\usepackage[colorlinks,linkcolor={black},citecolor={blue!80!black},urlcolor={blue!80!black}]{hyperref}

\title{\textbf{\textsc{Gate Question.35}}}
\author{\textbf{\textit{\teflipflopxtbf{Meet Patel(MIT WPU )}}}}
\begin{document}

\date{}
\maketitle
\tableofcontents


\section{PROBLEM}
\textbf{(GATE CS-2006)}
\textbf{Q.35} Consider the circuit above. Which one of the following options correctly represents f (x, y, z)?

        

\begin{tikzpicture}[x=0.75pt,y=0.75pt,yscale=-1,xscale=1]
%uncomment if require: \path (0,235); %set diagram left start at 0, and has height of 235

%Rounded Same Side Corner Rect [id:dp363116648723401] 
\draw   (183.43,18.38) .. controls (187.38,18.55) and (190.61,21.9) .. (190.64,25.85) -- (190.95,74.84) .. controls (190.98,78.79) and (187.79,81.86) .. (183.84,81.69) -- (155.18,80.44) .. controls (155.18,80.44) and (155.18,80.44) .. (155.18,80.44) -- (154.77,17.13) .. controls (154.77,17.13) and (154.77,17.13) .. (154.77,17.13) -- cycle ;
%Straight Lines [id:da5097704672655627] 
\draw    (103,27) -- (154,27) ;
%Straight Lines [id:da30040604368415424] 
\draw    (100.67,63) -- (154.67,63) ;
%Straight Lines [id:da949301259106712] 
\draw    (100,113) -- (164.67,113) ;
%Straight Lines [id:da614604161737778] 
\draw    (165.33,81) -- (164.67,113) ;
%Straight Lines [id:da6917161344721869] 
\draw    (190.33,51) -- (270.67,51) ;
%Straight Lines [id:da43350056257805236] 
\draw    (270.67,51) -- (270.67,82.33) ;
%Rounded Same Side Corner Rect [id:dp5971787189424798] 
\draw   (359.59,65.2) .. controls (364.01,65.14) and (367.63,68.68) .. (367.68,73.1) -- (368.31,127.1) .. controls (368.37,131.52) and (364.83,135.14) .. (360.41,135.19) -- (328.41,135.56) .. controls (328.41,135.56) and (328.41,135.56) .. (328.41,135.56) -- (327.59,65.57) .. controls (327.59,65.57) and (327.59,65.57) .. (327.59,65.57) -- cycle ;
%Straight Lines [id:da49677885025359036] 
\draw    (270.67,82.33) -- (328,82.33) ;
%Straight Lines [id:da9441929292086915] 
\draw    (254.67,115) -- (328.67,114.33) ;
%Straight Lines [id:da3423958793932156] 
\draw    (368.67,99.67) -- (428.33,100.33) ;
%Straight Lines [id:da5414722703413462] 
\draw    (346.67,136) -- (347.67,172.33) ;
%Straight Lines [id:da6523992677426826] 
\draw    (257.67,172.33) -- (347.67,172.33) ;

% Text Node
\draw (156.77,20.13) node [anchor=north west][inner sep=0.75pt]   [align=left] {0};
% Text Node
\draw (156,52) node [anchor=north west][inner sep=0.75pt]   [align=left] {1};
% Text Node
\draw (92,20) node [anchor=north west][inner sep=0.75pt]   [align=left] {x};
% Text Node
\draw (98.67,59.33) node   [align=left] {\begin{minipage}[lt]{10.88pt}\setlength\topsep{0pt}
y'
\end{minipage}};
% Text Node
\draw (92.31,105.7) node  [rotate=-358.5] [align=left] {\begin{minipage}[lt]{10.2pt}\setlength\topsep{0pt}
z
\end{minipage}};
% Text Node
\draw (208.5,28.33) node   [align=left] {\begin{minipage}[lt]{26.52pt}\setlength\topsep{0pt}
MUX
\end{minipage}};
% Text Node
\draw (333.5,80.67) node   [align=left] {\begin{minipage}[lt]{9.75pt}\setlength\topsep{0pt}
0
\end{minipage}};
% Text Node
\draw (250.5,111) node   [align=left] {\begin{minipage}[lt]{8.67pt}\setlength\topsep{0pt}
x
\end{minipage}};
% Text Node
\draw (333.83,112.5) node   [align=left] {\begin{minipage}[lt]{9.29pt}\setlength\topsep{0pt}
1
\end{minipage}};
% Text Node
\draw (257.67,172.33) node   [align=left] {\begin{minipage}[lt]{9.29pt}\setlength\topsep{0pt}
y
\end{minipage}};
% Text Node
\draw (418,78.73) node [anchor=north west][inner sep=0.75pt]    {$f$};


\end{tikzpicture}





\begin{enumerate}[label=(\Alph*)]
	\item $ xz'+ xy + y'z$
	\item $ xz' + xy + (yz)'$
	\item $ xz + xy + (yz)'$
	\item $ xz + xy' + y'z$
\end{enumerate}
\bigskip

\section{COMPONENTS}
	\begin{tabularx}{0.45\textwidth} {  
  | >{\centering\arraybackslash}X  
  | >{\centering\arraybackslash}X  
  | >{\centering\arraybackslash}X | } 
\hline 
\textbf{Component} &  \textbf{Value} & \textbf{Quantity}\\ 
\hline 
Arduino & UNO & 1 \\   
\hline 
Bread board & - & 1 \\ 
\hline
Jumper wires & M-M & 20 \\ 
\hline 
LED & - & 1\\ 
\hline 
\end{tabularx}
\bigskip



\section{TRUTH TABLE}
The Truth Table for the identities got after the MUX is as follows:
F = xz' + xy + y'z
\begin{table}[ht!]
	\centering
\begin{tabular}{ |c |c |c |c |c |c |} 
\hline 
\newline 
	\textbf{x} & \textbf{y} & \textbf{z} & \textbf{F} \\ 
\hline  
	0 & 0 & 0 &0 \\   
	0 & 0 & 1 &1 \\  
	0 & 1 & 0 &1 \\  
	0 & 1 & 1 &0 \\  
	1 & 0 & 0 &1 \\  
	1 & 0 & 1 &0 \\  
	1 & 1 & 0 &0 \\  
	1 & 1 & 1 &1 \\  
\hline 
\end{tabular}
	\caption{}
\end{table}

\bigskip


\paragraph{}
	Here, Except \textbf{(A)} identity all other identies are valid according to the mentioned Equation .
\end{enumerate}
\bigskip

\section{ARDUINO CONNECTIONS}

1) The connections between ARDUINO and LED are as follows:
\begin{table}[ht!] 
    \centering 
    \begin{tabular}{|c|c|c|c|c|c|c|c|} 
    \hline 
       ARDUINO& $ 2$&$ grnd $\\ 
    \hline 
        LED& anode&cathode \\ 
    \hline 
    \end{tabular} 
    \caption{} 
\end{table} 
\\

2) The connections between Inputs(x,y,z) and Arduino are as fllows:
\begin{table}[ht!] 
    \centering 
    \begin{tabular}{|c|c|c|c|c|} 
    \hline 
        Inputs& $ x$ &$ y$&$ z$ \\ 
         \hline 
         ARDUINO& $ 6$& $ 7$& $ 8$ \\ 
         \hline 
    \end{tabular} 
\caption{} 
\end{table} 
\\

3) The inputs \textbf{x,y,z} here are connected to Arduino D6,D7,D8 pins.\\

4) The values for these inputs are conncted either to GND or 5V according to the truth table.\\
\section{CODE}
\paragraph{}
	The arduino code can be downloaded from the below link.
\begin{center} 
\fbox{\parbox{8.5cm}{\url{https://github.com/meet-304/meet-304/tree/main/Assignment/codes/src/Aurduino.cpp}}} 
\end{center}


\end{document}
